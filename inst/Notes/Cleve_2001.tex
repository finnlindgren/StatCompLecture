\documentclass[a4paper,twocolumn,11pt]{article}

\usepackage{times}
\usepackage{verbatim}
\usepackage{subfigure}
\usepackage{graphicx}
\usepackage{stfloats}
\usepackage[colorlinks]{hyperref}

\newcommand{\Matlab}{{\textsc{Matlab}}}

\title{MATLAB News \& Notes - Spring 2001: 
Cleve's Corner\\
Normal Behavior:
Ziggurat algorithm generates normally distributed random numbers}
\author{Cleve Moler\\
\footnotesize\url{http://www.mathworks.com/company/newsletters/news_notes/clevescorner/spring01_cleve.html}}
\date{}

\begin{document}
\maketitle

{\Matlab}'s function \verb!rand(m,n)! generates an $m$-by-$n$ matrix
with elements drawn from a uniform distribution, while
\verb!randn(m,n)!  generates a matrix with normally or Gaussian
distributed elements. We described the algorithm that {\Matlab} uses for
uniform distributions five years ago in this newsletter\footnote{\url{http://www.mathworks.com/company/newsletters/news_notes/pdf/Cleve.pdf}}. Now
it's time to describe algorithms for normal distributions.

Almost all algorithms for generating normally distributed random
numbers are based on transformations of uniform distributions. The
simplest way to generate an $m$-by-$n$ matrix with approximately
normally distributed elements is to use the expression
$$
\verb!sum(rand(m,n,12),3) - 6!
$$ This works because \verb!R = rand(m,n,p)! generates a
three-dimensional uniformly distributed array and \verb!sum(R,3)! sums
along the third dimension. The result is a two-dimensional array with
elements drawn from a distribution with mean $p/2$ and variance $p/12$
that approaches a normal distribution as $p$ increases. If we take $p =
12$, we get a pretty good approximation to the normal distribution and
we get the variance to be equal to one without any additional
scaling. There are two difficulties with this approach. It requires
twelve uniforms to generate one normal, so it is slow. And the finite
$p$ approximation causes it to have poor behavior in the tails of the
distribution.

Older versions of {\Matlab} - before {\Matlab}~5 - used the \emph{polar}
algorithm. This generates two values at a time. It involves finding a
random point in the unit circle by generating uniformly distributed
points in the $[0,1]\times[0,1]$ square and rejecting any points
outside of the circle. Points in the square are represented by vectors
with two components. The rejection portion of the code is:
\begin{verbatim}r = 2;
while r > 1
   u = 2*rand(2,1)-1
   r = u'*u
end
\end{verbatim}
For each point accepted, the polar transformation
\begin{verbatim}v = sqrt(-2*log(r)/r)*u
\end{verbatim}
produces a vector with two independent normally distributed
elements. This algorithm does not involve any approximations, so it
has the proper behavior in the tails of the distribution. But it is
moderately expensive. Over 21\% of the uniform numbers are rejected
when they fall outside of the circle and the square root and logarithm
calculations contribute significantly to the cost.

Beginning with {\Matlab}~5, and continuing with {\Matlab}~6, \verb!randn!
uses a sophisticated table lookup algorithm developed by George
Marsaglia of Florida State University. Marsaglia calls his approach
the "ziggurat" algorithm. Ziggurats are ancient Mesopotamian terraced
temple mounds that, mathematically, are two-dimensional step
functions. A one-dimensional ziggurat underlies Marsaglia's algorithm.
\begin{figure*}[t]			
\centering
\subfigure[{The ziggurats of ancient Mesopotamia are all in ruins
    today, but the Mayan pyramid of Kukulkan illustrates the same
    step-function structure.}]{ \includegraphics[width=0.45\linewidth]
  {figs/spring01_pyramid}
%   \label{fig:subfig1}
 }
~
 \subfigure[{Step-function, or ziggurat, covering of the probability
     density function for the normal distribution.}]{
   \includegraphics[width=0.45\linewidth] {figs/spring01_step}
%   \label{fig:subfig2}
 }
%\label{myfigure}
%\caption{Global figure caption}
\end{figure*}			

Marsaglia has refined his ziggurat algorithm over the years. An early
version is described in Knuth's classic \emph{The Art of Computer
  Programming}, volume~II. {\Matlab}'s version is described by Marsaglia
and W.~W.~Tsang in the \emph{SIAM Journal of Scientific and
  Statistical Programming}, volume 5, 1984. A Fortran subroutine is
discussed briefly in the Kahaner, Moler, and Nash text book, Numerical
Methods and
Software\footnote{\url{http://www.mathworks.com/moler/ncm/drnor.f}},
and is available from the MathWorks FTP site. A recent version is
available in the online electronic journal, \emph{Journal of
  Statistical
  Software}\footnote{\url{http://www.jstatsoft.org/v05/i08}}. We
describe this recent version here because it is the most elegant. The
version actually used in {\Matlab} is more complicated, but is based on
the same ideas and is just as effective.

The probability density function, or pdf, of the normal distribution is the bell-shaped curve
$$
f(x) = c\,\exp(-x^2/2)
$$ where $c=1/(2\pi)^{1/2}$ is a normalizing constant that we can
ignore. If we generate random points $(x,y)$, uniformly distributed in
the plane, and reject all of them that do not fall under this curve,
the remaining $x$'s form our desired normal distribution.

The ziggurat algorithm covers the area under the pdf by a slightly
larger area with $n$ sections. The accompanying picture has $n = 8$;
the actual code uses $n = 128$. The ziggurat is shown in solid
blue. The top $n-1$ sections are rectangles. The bottom section is a
rectangle together with an infinite tail under the graph of
$f(x)$. The right-hand edges of the rectangles are at the points
$x_k$, $k=2,\ldots,n$, shown with circles in the picture. With
$f(x_1)=1$ and $f(x_{n+1})=0$, the height of the k-th section is
$f(x_k)-f(x_{k+1})$. The key idea is to choose the $x_k$'s so that all
$n$ sections, including the unbounded one on the bottom, have the same
area. There are other algorithms that approximate the area under the
pdf with rectangles. The distinguishing features of Marsaglia's
algorithm are the facts that the rectangles are horizontal and have
equal areas.

For a specified number, $n$, of sections, it is possible to solve a
transcendental equation to find $x_n$, the point where the infinite
tail meets the first rectangular section. In our picture with $n = 8$,
it turns out that $x_n = 2.34$. In the actual code with $n = 128$,
$x_n= 3.4426$. Once $x_n$ is known, it is easy to compute the common
area of the sections and the other right-hand end points, $x_k$. It is
also possible to compute $\sigma_k=x_{k-1}/x_{k}$, the fraction of
each section that lies underneath the section above it. Let's call
these fractional sections the "core" of the ziggurat. The right-hand
edge of the core is the dotted blue line in our picture. The
computation of these $x_k$'s and $\sigma_k$'s is done in an
initialization section of code that is run only once in any
{{\Matlab}} session.

After the initialization, normally distributed random numbers can be
computed very quickly. The key portion of the code computes a single
random integer, $k$, between $1$ and $n$, and a single uniformly
distributed random number, $u$, between $-1$ and $1$. A check is then
made to see if $u$ falls in the core of the $k$-th section. If it
does, then we know that $ux_k$ is the $x$-coordinate of a point under
the pdf and this value can be returned as one sample from the normal
distribution. The code looks something like this. (In the actual
built-in function, a fast shift-register generator computes the random
integer.)
\begin{verbatim}k = ceil(128*rand);
u = 2*rand-1;
if abs(u) < sigma(k)
   v = u*x(k);
   return
end
\end{verbatim}

Most of the $\sigma_k$'s are greater than $0.98$, and the test is true
over 97\% of the time. One normal random number can usually be
computed from one random integer, one random uniform, an if-test and a
multiplication. No square roots or logarithms are required.

The point determined by $k$ and $u$ will fall outside of the core less
than 3\% of the time. This happens when $k= 1$ because the top section
has no core, when $k$ is between $2$ and $n-1$ and the random point is
in one of the little rectangles covering the graph of $f(x)$, or when
$k = n$ and the point is in the infinite tail. In these cases,
additional computations involving logarithms, exponentials and more
uniform samples are required.

It is important to realize that, even though the ziggurat step
function only approximates the probability density function, the
resulting distribution is exactly normal. Decreasing $n$ decreases the
amount of storage required for the tables and increases the fraction
of time that extra computation is required, but does not affect the
accuracy. Even with $n = 8$, we would have to do the more costly
corrections almost 23\% of the time, instead of less than 3\%, but we
would still get an exact normal distribution.

With this algorithm, {\Matlab}~6 can generate normally distributed
random numbers as fast as it can generate uniformly distributed
ones. In fact, {\Matlab} on my 800~MHz~Pentium~III laptop can generate
over 10~million random numbers from either distribution in less than
one second.

{\Matlab}'s built-in function \verb!randn! uses two unsigned 32-bit
integers, one for the seed in the uniform generator and one as the
shift register for the integer generator. The code that processes
points outside the core of the ziggurat accesses these two generators
independently, so the period of the overall generator is $2^{64}$. At
10 million normal deviates per second, \verb!randn! will take over
58,000 years before it repeats.

\end{document}
